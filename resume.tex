%realize  LaTeX file for resume
% This file uses the resume document class (res.cls)

\documentclass{res}
%\usepackage{helvetica} % uses helvetica postscript font (download helvetica.sty)
%\usepackage{newcent}   % uses new century schoolbook postscript font
\usepackage{url}
\usepackage{enumitem}
\setlength{\textheight}{10in} % increase text height to fit on 1-page

\usepackage[T1]{fontenc}
\usepackage[utf8]{inputenc}
\usepackage{tgbonum}


\begin{document}
\name{JINYANG YAO\\[12pt]}     % the \\[12pt] adds a blank
				        % line after name

\address{\textbf{Email:} jimmy123good@hotmail.com \\
        \textbf{Github: }https://github.com/ailrk \\
         \textbf{Blog:}    https://ailrk.github.io/home \\
         \textbf{Linkedin:} https://www.linkedin.com/in/jinyang-yao-649549208 \\
         \textbf{Contact:}  +1 (250) 899 2600}
\begin{resume}

\section{EDUCATION}
The University of British Columbia, Kelowna, BC, Canada  \hspace{1.2in} \textbf{2017-2021}\\
    Honours in Computer Science, minor in Mathematics. Average 85.9/100. \\

\vspace{-0.2in}
\section{PAST EXPERIENCES}
\begin{itemize}[leftmargin=-.1in]
    \item \textbf{2019 - 2020:} Chongqing University HVAC department
      \vspace{0.05in}\\ Participated in the ``Yangzi River area air conditioning and heating solution and its corresponding systems'' research project. Developed a program using NSGAII genetic algorithm to select the optimal set of building designs that satisfy multi-objective constraints.

      Designed and implemented the project's internal web platform. The platform collects, organizes, and displays data from temperature sensors installed across 5 provinces.

     \\
    \item \textbf{2020 - 2021: } UBC computer science honor program. \\
        Using semi-supervised learning technique to assess programming assignments. Developed an iterative method and the corresponding system that allows human intervention to guide the algorithm to improve the clustering result. \\

\end{itemize}

\vspace{-0.2in}
\section{PERSONAL PROJECTS}

    \vspace{-0.1in}
    \begin{tabbing}
        \hspace{2.4in}\= \hspace{0.9in}\= \kill
        {\bf cppparsec } \>               \>\url{github.com/ailrk/cppparsec}\\

    \end{tabbing}\vspace{-30pt}
    A C++ monadic parser combinator inspired by Haskell's parsec library.

    \vspace{-0.2in}
    \begin{tabbing}
        \hspace{2.4in}\= \hspace{0.9in}\= \kill
        {\bf tml} \>               \>\url{github.com/ailrk/tml}\\

    \end{tabbing}\vspace{-30pt}
    Template meta language. A ml style functional programming language that runs in C++ template.

    \begin{tabbing}
        \hspace{2.4in}\= \hspace{0.9in}\= \kill
        {\bf pogger } \>               \>\url{github.com/ailrk/pogger}\\

    \end{tabbing}\vspace{-30pt}
    A partially R5S5 compliant scheme implementation in Haskell.


\section{SKILLS}
    \begin{itemize}[leftmargin=-.2in]
        \setlength\itemsep{-1em}
          \item \textbf{Programming Language:} Mostly used: Haskell, C++,  Python, Typescript. Worked with: C\#, Java. Familiar with: Common lisp, Rust, Ocaml.
          \\

        \item \textbf{C++: 3 years}
          Familiar with the C++ memory model, template meta programming,
          generic programming and some implementation details of the STL.
          \\

        \item \textbf{Haskell 2 years}
          Familiar with functional programming.
          Understand GHC compilation process and some GHC optimizations.
          Familiar with type level programming.
          \\

        \item \textbf{Python: 5 years}
          Familiar with flask web framework, sqlalchemy, and
          traditional machine learning algorithms with sklearn.
          \\

        \item \textbf{Typescript: 2 years:}
          Understand the structural typing system and some type level techniques.
          Familiar with libuv and event loop system. Understand the reactor pattern.
          Familiar with React and nodeJS.
          \\

        \item \textbf{Compiler} Familiar with traditional compilation techniques, optimizations and some program analyses. Familiar with various parsing techniques.
          \\

        \item \textbf{Math:}
          Minor in Mathematics.
          Linear Programming, Dynamic system, Number Theory, Abstract Algebra and their applications in error correction.  \\
        \item \textbf{Language:} Fluent in \textbf{English}. Native \textbf{Chinese} speaker. Learning \textbf{Japanese}.
    \end{itemize}

\end{resume}
\end{document}
